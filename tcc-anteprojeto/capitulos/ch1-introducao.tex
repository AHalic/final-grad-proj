% ==============================================================================
% TCC - Nome do Aluno
% Capítulo 1 - Introdução
% ==============================================================================
\chapter{Introdução}
\label{sec-intro}

O advento da Internet e da \textit{World Wide Web} (WWW), na década de 80, trouxe uma nova
forma de comunicação e interação entre as pessoas. O crescimento da \textit{Web} ocorreu de forma tão
rápida que, em pouco tempo, ela se tornou uma plataforma essencial para os negócios, comércios
e outros setores da sociedade. Nesse contexto, emerge na \textit{Web} uma variedade de aplicações
complexas e de grande porte, os chamados \textit{WebApps}, que no entanto, não eram desenvolvidas com o apoio de
metodologias e processos bem definidos.

Surge então a necessidade de adaptar métodos da Engenharia de Software ao desenvolvimento
das WebApps, de forma a construir soluções eficazes e garantir a qualidade e a manutenibilidade 
dos sistemas \cite{beder:2017}. A Engenharia \textit{Web} pode ser definida como o uso de 
princípios científicos, de engenharia, de gerência e abordagens sistemáticas para o desenvolvimento,
implantação e manutenção de aplicações \textit{Web} de alta qualidade \cite{murugesan:2001}.

Ao final dos anos 90 a ideia de \textit{frameworks web} começou a ser popularizada, reunindo 
várias bibliotecas úteis para desenvolvimento \textit{web} em uma única \textit{stack} de \textit{software}
para os desenvolvedores utilizarem e agilizarem o processo de desenvolvimento. A maioria dos
\textit{frameworks Web} são baseados na arquitetura Model View Controller (MVC), alguns exemplos
são o Ruby on Rails\footnote{\url{https://rubyonrails.org}}, Django\footnote{\url{https://www.djangoproject.com}}, 
Laravel\footnote{\url{https://laravel.com}}, Spring MVC\footnote{\url{https://spring.io}}, entre outros.

Apesar do crescimento no uso dos \textit{frameworks Web}, não havia um método da Engenharia \textit{Web} 
voltado exclusivamente para o desenvolvimento de sistemas que os utilizassem, nesse contexto, em sua 
tese de mestrado, \citeonline{souza:2007} propos o FrameWeb, método para o projeto de sistemas de 
informação \textit{Web} baseado em \textit{frameworks}. Objetivando o aumento da produtividade da 
equipe de desenvolvimento.

Neste trabalho, será aplicado o método FrameWeb em uma nova implementação do SCAP (Sistema de Controle de Afastamento de Professores), 
\textit{WebApp} que auxilia o controle de afastamento de professores do Departamento de Informática (DI) 
pela Internet. Para isso, será utilizado o \textit{framework} Next.js, que diferente dos trabalhos anteriores, 
não é um \textit{framework} baseado em MVC.


\section{Motivação e Justificativa}
\label{sec-intro-motjus}

O sistema SCAP surge da necessidade de facilitar e agilizar o processo de requisição de afastamento de 
professores do Departamento de Informática da UFES. Originalmente, o processo é realizado 
de forma manual, com o envio de uma série de e-mails, o que faz com que este seja um processo lento.

Em outros trabalhos este sistema foi implementado utilizando diferentes \textit{frameworks} e o método FrameWeb, a fim de analisar a aplicação do método em diferentes contextos, são exemplos: \citeonline{duarte:2014} com Java EE 7, ~\citeonline{prado:2015} com VRaptor 4 e ~\citeonline{gomes:2022} com Angular. 
Este trabalho tem como motivação utilizar o \textit{framework} Next.js, seguindo o método FrameWeb~\cite{souza:2007}. Com a finalidade de contribuir para a avaliação do método com um novo conjunto de \textit{frameworks}.


%%% Início de seção. %%%
\section{Objetivos}
\label{sec-intro-obj}

Este trabalho tem como objetivo geral aplicar o método FrameWeb~\cite{souza:2007} em uma nova implementação do SCAP (Sistema de Controle de Afastamento de Professores),
baseando-se nos requisitos levantados por \cite{duarte:2014} e \cite{prado:2015}. Será utilizado o \textit{framework} Next.js, de forma a contribuir com a análise do método FrameWeb, assim como em sua evolução.

Para isso, faz-se necessária a definição de objetivos específicos que, juntos, auxiliam na conclusão do objetivo geral, sendo eles:
\begin{itemize}
    \item Compreender o método FrameWeb;
    \item Analisar os requisitos da aplicação SCAP;
    \item Implementar a aplicação SCAP utilizando o \textit{framework} Next.js e o método FrameWeb.
\end{itemize}


%%% Início de seção. %%%
\section{Método de Desenvolvimento do Trabalho}
\label{sec-intro-met}

Para que os objetivos apresentados na seção anterior sejam satisfeitos, os seguintes passos devem ser seguidos:

\begin{itemize}
    \item Conduzir estudo abrangente da bibliografia disponível sobre o FrameWeb~\cite{souza:2007,souza:2020};
    \item Revisar e compreender os requisitos da aplicação SCAP levantados por \citeonline{duarte:2014} e \citeonline{prado:2015};
    \item Estudar padrões de arquitetura, em específico MVC e MVVM, assim como os \textit{frameworks} baseados em tais padrões, e \textit{frameworks} SPA;
    \item Elaborar os modelos FrameWeb e gerar o Documento de Projeto, considerando o \textit{frameworks} escolhido para a implementação;
    \item Implementar o sistema SCAP, a partir dos requisitos já levantados e do Documento de Projeto gerado, utilizando o \textit{framework} Next.js;
    \item Redigir a monografia utilizando o \textit{template} abnTeX\footnote{\url{https://www.abntex.net.br/}} para a escrita em \latex\footnote{\url{https://www.latex-project.org/}} seguindo os requisitos das normas da ABNT (Associação Brasileira de Normas Técnicas);
\end{itemize}


%%% Início de seção. %%%
\section{Cronograma}
\label{sec-intro-crono}

\begin{itemize}
\item Atividade 1: Revisão bibliográfica sobre o método FrameWeb e \textit{frameworks} baseados em MVC e MVVM;
\item Atividade 2: Estudo dos requisitos da aplicação SCAP;
\item Atividade 3: Escrita do anteprojeto a partir dos resultados obtidos nas atividades 1 e 2;
\item Atividade 4: Produção do documento de Projeto, segundo método FrameWeb;
\item Atividade 5: Desenvolvimento da aplicação SCAP com o \textit{framework} Next.js;
\item Atividade 6: Escrita do Trabalho de Conclusão de Curso, apresentando os resultados obtidos e a produção desenvolvida;
\item Atividade 7: Apresentação do trabalho para a banca avaliadora.
\end{itemize}

\begin{table}[h]
\centering
\begin{tabular}{c|c|c|c|c|c|}
\cline{2-6}
\multicolumn{1}{l|}{} & \textbf{Agosto} & \textbf{Setembro} & \textbf{Outubro} & \textbf{Novembro} & \textbf{Dezembro} \\ \hline
\multicolumn{1}{|c|}{\textbf{Atividade 1}} & x &   &   &   &   \\ \hline
\multicolumn{1}{|c|}{\textbf{Atividade 2}} & x &   &   &   &   \\ \hline
\multicolumn{1}{|c|}{\textbf{Atividade 3}} & x & x &   &   &   \\ \hline
\multicolumn{1}{|c|}{\textbf{Atividade 4}} &   & x & x &   &   \\ \hline
\multicolumn{1}{|c|}{\textbf{Atividade 5}} &   &   & x & x &   \\ \hline
\multicolumn{1}{|c|}{\textbf{Atividade 6}} &   &   & x & x & x \\ \hline
\multicolumn{1}{|c|}{\textbf{Atividade 7}} &   &   &   &   & x \\ \hline
\end{tabular}
\end{table}