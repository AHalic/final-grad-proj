% ==============================================================================
% TCC - Nome do Aluno
% Capítulo 1 - Introdução
% ==============================================================================
\chapter{Introdução}
\label{sec-intro}

%\hl{Texto.}

%\hrulefill

A \textbf{Introdução} deve conter de \textbf{3 a 5 páginas}. Primeiramente, deve ser colocada a
Descrição do trabalho, a qual apresenta o contexto do trabalho e a definição do escopo do
mesmo. Deve-se delimitar o escopo do trabalho de forma que haja condições
técnicas suficientes para que o mesmo seja concluído em tempo hábil.


%%% Início de seção. %%%
\section{Motivação e Justificativa}
\label{sec-intro-motjus}

A \textbf{Motivação} apresenta as circunstancias que interferiram na escolha do tema.
A \textbf{Justificativa} apresenta o porquê da escolha do tema, o problema a ser resolvido
e a relevância do trabalho, referindo-se a estudos anteriores sobre o tema, ressaltando
suas eventuais limitações e destacando a necessidade de se continuar pesquisando o
assunto.

%%% Início de seção. %%%
\section{Objetivos}
\label{sec-intro-obj}

% Nesta subseção, deve ser descrito o \textbf{objetivo geral} do trabalho, detalhando em
% seguida, seus \textbf{objetivos específicos}.

% O \textbf{Objetivo Geral} expressa a finalidade do trabalho: para quê? Deve ter coerência
% direta com o tema do trabalho e ser apresentado em uma frase que inicie com um verbo
% no infinitivo. O objetivo geral do trabalho está relacionado ao resultado principal do trabalho.

% Os \textbf{Objetivos Específicos} apresentam os detalhes e/ou desdobramentos do
% objetivo geral que levam a resultados intermediários e relevantes para alcançar o objetivo geral. Sempre será mais de um objetivo específico, todos iniciando com verbo no infinitivo.

Este trabalho tem como objetivo geral aplicar o método FrameWeb~\cite{souza:2007} em uma nova implementação do SCAP (Sistema de Controle de Afastamento de Professores),
baseando-se nos requisitos levantados por \cite{duarte:2014} e \cite{prado:2015}. Será utilizado o \textit{framework} Next.js, de forma a contribuir com a análise do método FrameWeb, assim como em sua evolução.

Para isso, faz-se necessária a definição de objetivos específicos que, juntos, auxiliam na conclusão do objetivo geral, sendo eles:
\begin{itemize}
    \item Compreender o método FrameWeb;
    \item Analisar os requisitos da aplicação SCAP;
    \item Implementar a aplicação SCAP utilizando o \textit{framework} Next.js e o método FrameWeb.
\end{itemize}


%%% Início de seção. %%%
\section{Método de Desenvolvimento do Trabalho}
\label{sec-intro-met}
\sophie{o método deveria ser mais próximo a um texto descritivo?}

% Nesta subseção, deve ser apresentado o \textbf{Método de Desenvolvimento} (ou o \textbf{Método de Pesquisa}, quando for o caso) do trabalho. Aqui são apresentados os procedimentos/técnicas que serão usados durante o desenvolvimento do trabalho. 
Para que os objetivos apresentados na seção anterior sejam satisfeitos, os seguintes passos devem ser seguidos:

\begin{itemize}
    \item Conduzir estudo abrangente da bibliografia disponível sobre o FrameWeb~\cite{souza:2007,souza:2020};
    \item Revisar e compreender os requisitos da aplicação SCAP levantados por \cite{duarte:2014} e \cite{prado:2015};
    \item Estudar padrões de arquitetura, em específico MVC e MVVM, assim como os \textit{frameworks} baseados em tais padrões;
    \item Elaborar os modelos FrameWeb e gerar o Documento de Projeto, considerando o \textit{frameworks} escolhido para a implementação;
    \item Implementar o sistema SCAP, a partir dos requisitos já levantados e do Documento de Projeto gerado, utilizando o \textit{framework} Next.js;
    \item Redigir a monografia utilizando o \textit{template} abnTeX\footnote{\url{https://www.abntex.net.br/}} para a escrita em \latex\footnote{\url{https://www.latex-project.org/}} seguindo os requisitos das normas da ABNT (Associação Brasileira de Normas Técnicas);
\end{itemize}


%%% Início de seção. %%%
\section{Cronograma}
\label{sec-intro-crono}

% O \textbf{Cronograma de Execução} apresenta a distribuição no tempo das atividades
% que deverão ser desenvolvidas ao longo do trabalho. As atividades apresentadas no cronograma devem estar alinhadas com os objetivos apresentados na Subseção~\ref{sec-intro-obj}, ou seja, elas devem ser capazes de produzir os resultados necessários para alcançar os objetivos estabelecidos. Deve-se apresentar uma listagem com a descrição de cada uma dessas atividades e em seguida mostrada uma tabela contendo todas as atividades previstas juntamente com a  previsão do período de execução de cada uma.

\begin{itemize}
\item Atividade 1: Revisão bibliográfica sobre o método FrameWeb e \textit{frameworks} baseados em MVC e MVVM;
\item Atividade 2: Estudo dos requisitos da aplicação SCAP;
\item Atividade 3: Escrita do anteprojeto a partir dos resultados obtidos nas atividades 1 e 2;
\item Atividade 4: Produção do documento de Projeto, segundo método FrameWeb;
\item Atividade 5: Desenvolvimento da aplicação SCAP com o \textit{framework} Next.js;
\item Atividade 6: Escrita do Trabalho de Conclusão de Curso, apresentando os resultados obtidos e a produção desenvolvida;
\item Atividade 7: Apresentação do trabalho para a banca avaliadora.
\end{itemize}

\begin{table}[h]
\centering
\begin{tabular}{c|c|c|c|c|c|}
\cline{2-6}
\multicolumn{1}{l|}{} & \textbf{Agosto} & \textbf{Setembro} & \textbf{Outubro} & \textbf{Novembro} & \textbf{Dezembro} \\ \hline
\multicolumn{1}{|c|}{\textbf{Atividade 1}} & x &   &   &   &   \\ \hline
\multicolumn{1}{|c|}{\textbf{Atividade 2}} & x &   &   &   &   \\ \hline
\multicolumn{1}{|c|}{\textbf{Atividade 3}} & x & x &   &   &   \\ \hline
\multicolumn{1}{|c|}{\textbf{Atividade 4}} &   & x & x &   &   \\ \hline
\multicolumn{1}{|c|}{\textbf{Atividade 5}} &   &   & x & x &   \\ \hline
\multicolumn{1}{|c|}{\textbf{Atividade 6}} &   &   & x & x & x \\ \hline
\multicolumn{1}{|c|}{\textbf{Atividade 7}} &   &   &   &   & x \\ \hline
\end{tabular}
\end{table}