% ==============================================================================
% PG - Sophie Dilhon
% Capítulo 1 - Introdução
% ==============================================================================
\chapter{Introdução}
\label{chap-intro}

O advento da Internet e da \textit{World Wide Web} (WWW), na década de 80, trouxe uma nova
forma de comunicação e interação entre as pessoas. O crescimento da \textit{Web} ocorreu de forma tão
rápida que, em pouco tempo, ela se tornou uma plataforma essencial para os negócios, comércios
e outros setores da sociedade. Nesse contexto, emerge na \textit{Web} uma variedade de aplicações
complexas e de grande porte, os chamados \textit{WebApps}, que no entanto, não eram desenvolvidas com o apoio de
metodologias e processos bem definidos~\cite{murugesan:2001}.

Surge então a necessidade de adaptar métodos da Engenharia de Software ao desenvolvimento
das \textit{WebApps}, de forma a construir soluções eficazes e garantir a qualidade e a manutenibilidade 
dos sistemas~\cite{beder:2017}. A Engenharia \textit{Web} pode ser definida como o uso de 
princípios científicos, de engenharia, de gerência e abordagens sistemáticas para o desenvolvimento,
implantação e manutenção de aplicações \textit{Web} de alta qualidade~\cite{murugesan:2001}.

Ao final dos anos 90 a ideia de \textit{frameworks Web} começou a ser popularizada, reunindo 
várias bibliotecas úteis para desenvolvimento \textit{Web} em uma única \textit{stack} de \textit{software}
para os desenvolvedores utilizarem e agilizarem o processo de desenvolvimento. Muitos dos
\textit{frameworks Web} são baseados na arquitetura Model View Controller (MVC), alguns exemplos
são o Ruby on Rails,\footnote{\url{https://rubyonrails.org}} Django,\footnote{\url{https://www.djangoproject.com}} 
Laravel,\footnote{\url{https://laravel.com}} Spring MVC,\footnote{\url{https://spring.io}} entre outros.

Apesar do crescimento no uso dos \textit{frameworks Web}, não havia um método da Engenharia \textit{Web} 
voltado exclusivamente para o desenvolvimento de sistemas que os utilizassem, nesse contexto, em sua 
tese de mestrado, \citeonline{souza:2007} propôs o FrameWeb, método para o projeto de sistemas de 
informação \textit{Web} baseado em \textit{frameworks}. Objetivando o aumento da produtividade da 
equipe de desenvolvimento.

O método FrameWeb vem sendo evoluído nos últimos anos e, como parte de sua avaliação, diferentes implementações de um mesmo 
sistema de informação Web chamado SCAP (Sistema de Controle de Afastamento de Professores) --- uma \textit{WebApp} que auxilia o 
controle de afastamento de professores do Departamento de Informática (DI) pela Internet --- foram desenvolvidas, aplicando-se o método. 
Neste trabalho, será aplicado o método FrameWeb em uma nova implementação do SCAP, utilizando
a linguagem \textit{TypeScript} e o \textit{framework} Next.js que, diferente da grande maioria dos trabalhos anteriores, 
não é um \textit{framework} baseado em MVC.



%%% Início de seção. %%%
\section{Motivação e Justificativa}
\label{sec-intro-motjus}

Em seu trabalho de conclusão de curso, \citeonline{duarte:2014} desenvolveu a primeira implementação do SCAP, utilizando o método FrameWeb e os
\textit{frameworks} da plataforma Java EE 7, com o objetivo de avaliar o método com uma gama maior de \textit{frameworks} e tecnologias,
em relação à proposta original do método~\cite{souza:2007}. Em trabalhos seguintes, o método foi
avaliado com outros diferentes \textit{frameworks}, como o VRaptor 4~\cite{prado:2015}, 
Symfony~\cite{berger:2021}, Angular~\cite{gomes:2022}, etc.

Seguindo essa mesma linha, este trabalho tem como motivação avaliar o método FrameWeb com um novo \textit{framework},
o Next.js, que assim como Angular, é um \textit{framework} SPA (\textit{Single Page Application}), e com crescente popularidade 
no mercado. Contruibuindo então com a avaliação e evolução do método.

\patricia{Acho que poderia ser um pouco mais explícito aqui explicando 
como que as avaliações contribuem com a evolução do método. 
Talvez com um exemplo? Imagino que não seja algo empírico…
Existe alguma metodologia? Enfim, achei que isso ficou meio vago…}



%%% Início de seção. %%%
\section{Objetivos}
\label{sec-intro-obj}

Este trabalho tem como objetivo geral aplicar o método FrameWeb~\cite{souza:2007} em uma nova implementação do SCAP (Sistema de Controle de Afastamento de Professores),
baseando-se nos requisitos levantados por \citeonline{duarte:2014} e \citeonline{prado:2015} e utilizando o \textit{framework} Next.js, de forma a contribuir com a análise do método FrameWeb, assim como em sua evolução.

Para isso, faz-se necessária a definição de objetivos específicos que, juntos, auxiliam na conclusão do objetivo geral, sendo eles:
\begin{itemize}
    \item Compreender o método FrameWeb;
    \item Analisar os requisitos da aplicação SCAP;
    \item Implementar a aplicação SCAP utilizando o \textit{framework} Next.js e o método FrameWeb.
\end{itemize}


%%% Início de seção. %%%
\section{Método de Desenvolvimento do Trabalho}
\label{sec-intro-met}

Para que os objetivos apresentados na seção anterior sejam satisfeitos, os seguintes passos devem ser seguidos:

\begin{itemize}
    \item Conduzir estudo abrangente da bibliografia disponível sobre o FrameWeb~\cite{souza:2007,souza:2020};
    \item Revisar e compreender os requisitos da aplicação SCAP levantados por \citeonline{duarte:2014} e \citeonline{prado:2015};
    \item Estudar padrões de arquitetura, em específico MVC e MVVM, assim como os \textit{frameworks} baseados em tais padrões, e \textit{frameworks} SPA;
    \item Elaborar os modelos FrameWeb e gerar o Documento de Projeto, considerando o \textit{framework} escolhido para a implementação;
    \item Implementar o sistema SCAP, a partir dos requisitos já levantados e do Documento de Projeto gerado, utilizando o \textit{framework} Next.js;
    \item Redigir a monografia utilizando o \textit{template} abnTeX\footnote{\url{https://www.abntex.net.br/}} para a escrita em \latex\footnote{\url{https://www.latex-project.org/}} seguindo os requisitos das normas da ABNT (Associação Brasileira de Normas Técnicas).
\end{itemize}


%%% Início de seção. %%%
\section{Organização da Monografia}
\label{sec-intro-organizacao}

% \hl{Por fim, a última subseção da monografia apresenta a estrutura do texto. Por exemplo, para este documento esta seção poderia conter o seguinte texto:}

Além desta introdução, esta monografia é composta por outros cinco capítulos:

\begin{itemize}
	\item O Capítulo~\ref{chap-referencial} apresenta os aspectos relativos ao conteúdo teórico relevante para o trabalho;
	\item No Capítulo~\ref{chap-especificacao-requisitos} é descrito o sistema SCAP e seus requisitos;
	\item O Capítulo~\ref{chap-projeto} apresenta os modelos FrameWeb desenvolvidos e as tecnologias utilizadas na implementação, além de mostrar os resultados obtidos;
	\item O Capítulo~\ref{chap-conclusao}, por fim, discute os resultados obtidos e apresenta as conclusões do trabalho.
\end{itemize}


