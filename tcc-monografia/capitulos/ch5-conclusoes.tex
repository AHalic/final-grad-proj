% ==============================================================================
% PG - Nome do Aluno
% Capítulo 5 - Considerações Finais
% ==============================================================================
\chapter{Conclusão}
\label{chap-conclusao}

% \hl{Neste capítulo devem ser realizadas as considerações finais do trabalho, sendo apresentadas suas principais contribuições, limitações, lições aprendidas durante o desenvolvimento do trabalho, dificuldades enfrentadas e perspectivas de trabalhos futuros. O capítulo deve ter entre 3 e 5 páginas.}

Este capítulo apresenta as considerações finais do trabalho, dificuldades, e perspectivas para trabalhos futuros.
Na Seção~\ref{sec-conclusoes-consideracoes}, são apresentados as conclusões obtidas e suas relações com os objetivos definidos, enquanto
na Seção~\ref{sec-conclusoes-trabalhosfuturos} são apresentadas ideias e melhorias para trabalhos futuros.

%%% Início de seção. %%%
\section{Considerações Finais}
\label{sec-conclusoes-consideracoes}

% \hl{Esta seção deve apresentar um texto de fechamento do trabalho, devendo incluir considerações sobre o trabalho desenvolvido, suas limitações, contribuições, experiência adquirida pelo aluno e lições aprendidas ao longo do desenvolvimento, bem como dificuldades enfrentadas durante o desenvolvimento do trabalho. Nesta seção é preciso mostrar claramente a relação entre os resultados produzidos no trabalho e os objetivos estabelecidos no Capítulo}

% Objetivos
% • Compreender o método FrameWeb;
% • Analisar os requisitos da aplicação SCAP;
% • Implementar a aplicação SCAP utilizando o framework Next.js e o método FrameWeb

Neste projeto uma nova implementação do SCAP aplicando o método FrameWeb foi feita, dessa vez utilizando o \textit{framework} Next.js.
Os modelos FrameWeb produzidos foram essenciais para a implementação do SCAP, pois forneceram uma visão geral da aplicação, facilitando e agilizando o desenvolvimento do sistema.
Mesmo assim, ao longo da implementação os modelos foram revisitados diversas vezes, e inconsistências tiveram de ser consertadas.

Portanto, revisitando os objetivos do projeto, foi possível estudar e compreender o método FrameWeb para aplicá-lo na implementação do SCAP, considerando os requisitos levantados anteriormente.
Com isso, as disciplinas de Engenharia de Software foram revisitadas e colocadas em prática, e o conhecimento adquirido foi essencial para o andamento deste projeto.
Além disso, o projeto permitiu o aprendizado de um novo e moderno \textit{framework} de desenvolvimento, o Next.js.

\vitor{Dizer explicitamente acima se os objetivos do projeto foram satisfeitos completamente, parcialmente ou não foram satisfeitos.}

% Limitações
Uma limitação do método FrameWeb é o fato de este ter sido construído em cima de projetos Java, que não refletem a realidade do desenvolvimento de aplicações web modernas
SPA. Isso tornou a implementação mais complexa do que o esperado, uma vez que o Next.js não costuma seguir padrões como o \textit{DAO} ou ser Orientado a Objetos.
Tendo em vista que o foco do projeto era a aplicação e análise do método FrameWeb, duas funcionalidades não foram implementadas,
sendo elas: o envio de \textit{e-mails} e a geração de atas. Ambas funcionalidades são complexas e demandariam um tempo maior para serem implementadas.

\vitor{O trabalho do Pedro~\cite{hoppe:2023}, que você inclusive cita na Seção~\ref{sec-fundteo-frameweb}, visa trazer os frameworks SPA para o FrameWeb. O parágrafo acima não menciona isso. Elaborar melhor essa parte.}

Apesar do método FrameWeb utilizar UML, o que facilitaria a modelagem dos diagramas, a ferramenta utilizada para a modelagem, o \textit{Visual Paradigm}, 
tem uma curva de aprendizado acentuada, o que inicialmente dificultou a modelagem dos diagramas. Além disso, a ferramenta ainda não possui suporte para a geração de código
a partir dos diagramas, portanto o código foi implementado por inteiro do zero.

\vitor{A ferramenta possui suporte pra geração de código de alguns modelos (falta o de Navegação), mas para gerar código para o framework que você escolheu, é preciso construir os templates. Acertar o parágrafo acima para explicar de forma mais precisa porque não foi usada a geração de código.}


%%% Início de seção. %%%
\section{Trabalhos Futuros}
\label{sec-conclusoes-trabalhosfuturos}

% \hl{Nesta seção devem ser identificados trabalhos futuros que poderão ser realizados a partir dos resultados obtidos até o momento no trabalho. Idealmente, trabalhos futuros não devem apenas ser citados. Recomenda-se discutir aspectos sobre como podem ser realizados e por que é importante que sejam realizados (que benefícios podem ser obtidos com sua realização).}

Para trabalhos futuros, seria interessante analisar se os requisitos de fato contemplam as necessidades do departamento,
implementar as funcionalidades que ficaram pendentes, e fazer o \textit{deploy} da aplicação em um servidor.

Além disso, seria também interessante utilizar as ferramentas de modelagem, sejam elas o Editor FrameWeb~\cite{campos:2017} ou o \textit{Visual Paradigm} com o plugin desenvolvido~\cite{silva:2023},
com aplicações utilizando JavaScript, assim como testar o gerador de código.

A grande maioria das aplicações, dos trabalhos anteriores, foram feitas utilizando \textit{frameworks} Java ou JavaScript, e seria interessante
testar também a implementação com \textit{frameworks} Python,\footnote{Python, \url{https://www.python.org}} como Django, ou com \textit{frameworks} Ruby,\footnote{Ruby, \url{https://www.ruby-lang.org/en/}} como Ruby on Rails.
Por fim, para deixar o modelo mais robusto, é importante comparar todos os resultados obtidos ao longo de todas as diferentes implementações feitas.

\vitor{Acho que você poderia incluir aqui a tabela com os trabalhos anteriores que implementaram o SCAP, pegando o que o Danillo Gomes fez na Tabela 2 do TCC dele e atualizando com o seu trabalho. Isso mostraria que faltam implementações com Python e Ruby, como você diz no parágrafo. Você pode obter o código-fonte \LaTeX\ da tabela aqui: \url{https://bitbucket.org/vitorsouza-students/pg-2022-danilo-gomes/src/main/pg-monografia/capitulos/ch4-avaliacao.tex}. O TCC do Danillo em PDF está aqui: \url{http://www.inf.ufes.br/~vitorsouza/wp-content/papercite-data/pdf/gomes-pg22.pdf}}
