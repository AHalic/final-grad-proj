% ==============================================================================
% PG - Sophie Dilhon
% Capítulo 5 - Considerações Finais
% ==============================================================================
\chapter{Conclusão}
\label{chap-conclusao}

% \hl{Neste capítulo devem ser realizadas as considerações finais do trabalho, sendo apresentadas suas principais contribuições, limitações, lições aprendidas durante o desenvolvimento do trabalho, dificuldades enfrentadas e perspectivas de trabalhos futuros. O capítulo deve ter entre 3 e 5 páginas.}

Este capítulo apresenta as considerações finais do trabalho, dificuldades, e perspectivas para trabalhos futuros.
Na Seção~\ref{sec-conclusoes-consideracoes}, são apresentadas as conclusões obtidas e suas relações com os objetivos definidos, enquanto
na Seção~\ref{sec-conclusoes-trabalhosfuturos} são apresentadas ideias e melhorias para trabalhos futuros.

%%% Início de seção. %%%
\section{Considerações Finais}
\label{sec-conclusoes-consideracoes}

Neste projeto uma nova implementação do SCAP aplicando o método FrameWeb foi realizada, dessa vez utilizando o \textit{framework} Next.js.
Os modelos FrameWeb produzidos foram essenciais para a implementação do SCAP, pois forneceram uma visão geral da aplicação, facilitando e agilizando o desenvolvimento do sistema.
Mesmo assim, ao longo da implementação os modelos foram revisitados diversas vezes, e inconsistências tiveram de ser consertadas.

Os objetivos definidos no Capítulo~\ref{chap-intro} estão listados abaixo.
\begin{itemize}
    \item Compreender o método FrameWeb;
    \item Analisar os requisitos da aplicação SCAP;
    \item Implementar a aplicação SCAP utilizando o \textit{framework} Next.js e o método FrameWeb.
\end{itemize}

Foi possível estudar e compreender o método FrameWeb para aplicá-lo na implementação do SCAP, considerando os requisitos levantados anteriormente.
Com isso, as disciplinas de Engenharia de Software foram revisitadas e colocadas em prática, e o conhecimento adquirido foi essencial para o andamento deste projeto.
Além disso, o projeto permitiu o aprendizado de um novo e moderno \textit{framework} de desenvolvimento, o Next.js, por meio da implementação do sistema SCAP.
Assim, os objetivos foram totalmente satisfeitos.

\vitor{Dizer explicitamente acima se os objetivos do projeto foram satisfeitos completamente, parcialmente ou não foram satisfeitos.}


% Limitações
Em seu trabalho, \citeonline{hoppe:2023} fez adaptações do modelo FrameWeb para \textit{frameworks} SPA, o que facilitou 
a modelagem dos diagramas de navegação, principalmente. No entanto, algumas dificuldades apareceram ao longo da implementação,
isso pois foram encontradas poucas referências de projetos utilizando padrões, como \textit{Repository Pattern} ou DAO, com o \textit{framework} Next.js.
Em sua maioria o Next.js é utilizado apenas como \textit{frontend}, no entanto neste projeto ele foi utilizado como \textit{fullstack}.


Tendo em vista que o foco do projeto era a aplicação e análise do método FrameWeb, duas funcionalidades não foram implementadas,
sendo elas: o envio de \textit{e-mails} e a geração de atas. Ambas funcionalidades são complexas e demandariam um tempo maior para serem implementadas.

\vitor{O trabalho do Pedro~\cite{hoppe:2023}, que você inclusive cita na Seção~\ref{sec-fundteo-frameweb}, visa trazer os frameworks SPA para o FrameWeb. O parágrafo acima não menciona isso. Elaborar melhor essa parte.}

\sophie{
Fiquei em dúvida de como escrever essa parte, acho que seria mais uma dificuldade minha do que limitação do modelo, a minha dificuldade foi em relação
a eu não conhecer o Next e não ter achado muitas referencias de padrões de projeto usando ele como fullstack
(E eu não tinha muitas experiência implementando um back assim).

O trabalho do~\cite{hoppe:2023} me pareceu ter sido mais voltado para os modelos de navegação e nos frameworks SPA
usando orientação à objeto. No meu caso os components são na verdade funções e apesar de terem também um comportamento de Controller
ainda existe um controller no back-end,
revisando os meus modelos de navegação acho até que eles podem estar errados, já que por ser função não há uma chamada de método,
mas sim chamadas à api.
}


Apesar do método FrameWeb utilizar UML, o que facilitaria a modelagem dos diagramas, a ferramenta utilizada para a modelagem, o \textit{Visual Paradigm}, 
tem uma curva de aprendizado acentuada, o que inicialmente dificultou a modelagem dos diagramas. Além disso, a ferramenta possui suporte para a geração de código
a partir dos diagramas, porém houve uma grande dificuldade de usá-la, o que fez com que a implementação fosse feita manualmente.

\vitor{A ferramenta possui suporte pra geração de código de alguns modelos (falta o de Navegação), mas para gerar código para o framework que você escolheu, é preciso construir os templates. Acertar o parágrafo acima para explicar de forma mais precisa porque não foi usada a geração de código.}
\sophie{Implementei os templates mas não consegui gerar, achei que a documentação não estava completamente clara.
Posso escrever isso e dizer que um trabalho futuro seria melhorar a ferramenta e documentação?}


%%% Início de seção. %%%
\section{Trabalhos Futuros}
\label{sec-conclusoes-trabalhosfuturos}

% \hl{Nesta seção devem ser identificados trabalhos futuros que poderão ser realizados a partir dos resultados obtidos até o momento no trabalho. Idealmente, trabalhos futuros não devem apenas ser citados. Recomenda-se discutir aspectos sobre como podem ser realizados e por que é importante que sejam realizados (que benefícios podem ser obtidos com sua realização).}

Para trabalhos futuros, seria interessante analisar se os requisitos de fato contemplam as necessidades do departamento,
implementar as funcionalidades que ficaram pendentes, e fazer o \textit{deploy} da aplicação em um servidor.

Além disso, seria também interessante utilizar as ferramentas de modelagem, sejam elas o Editor FrameWeb~\cite{campos:2017} ou o \textit{Visual Paradigm} com o plugin desenvolvido~\cite{silva:2023},
com aplicações utilizando JavaScript, assim como testar o gerador de código.

A grande maioria das aplicações, dos trabalhos anteriores, apresentados na Tabela~\ref*{tbl-comparacao-trabalhos}, foram feitas utilizando \textit{frameworks} Java, JavaScript ou PHP. Portanto, para ampliar
o campo da pesquisa, seria interessante
implementar o sistema também com \textit{frameworks} Python,\footnote{Python, \url{https://www.python.org}} como Django, ou com \textit{frameworks} Ruby,\footnote{Ruby, \url{https://www.ruby-lang.org/en/}} como Ruby on Rails,
linguagens essas muito utilizados no mercado atual.
Por fim, para deixar o modelo mais robusto, é importante comparar todos os resultados obtidos ao longo de todas as diferentes implementações feitas.

\vitor{Acho que você poderia incluir aqui a tabela com os trabalhos anteriores que implementaram o SCAP, pegando o que o Danillo Gomes fez na Tabela 2 do TCC dele e atualizando com o seu trabalho. Isso mostraria que faltam implementações com Python e Ruby, como você diz no parágrafo. Você pode obter o código-fonte \LaTeX\ da tabela aqui: \url{https://bitbucket.org/vitorsouza-students/pg-2022-danilo-gomes/src/main/pg-monografia/capitulos/ch4-avaliacao.tex}. O TCC do Danillo em PDF está aqui: \url{http://www.inf.ufes.br/~vitorsouza/wp-content/papercite-data/pdf/gomes-pg22.pdf}}


\begin{table}[h]
	\caption{Listagem de implementações do SCAP.}
	\label{tbl-comparacao-trabalhos}
	\centering\def\tabularxcolumn#1{m{#1}}\def\arraystretch{1.0}
	\begin{tabularx}{\textwidth}{ 
			| >{\hsize=0.4\hsize}X
			| >{\hsize=0.5\hsize}X	
			| >{\hsize=0.7\hsize}X
			| >{\hsize=0.4\hsize}X
			|
		}
		\hline
		\textbf{Trabalho} & 
		\textbf{\textit{Framework} MVC} & 
		\textbf{Tecnologia ORM} &
		\textbf{Tecnologia DI} \\
		\hline
		
		\citeonline{duarte:2014} &
		JSF (Java) & 
		Hibernate (Data Mapper) &
		CDI \\ 
		\hline 
		
		\citeonline{prado:2015} & 
		VRaptor (Java) & 
		Hibernate (Data Mapper) &
		CDI \\
		\hline 
		
		\citeonline{pinheiro:2017} & 
		Laravel (PHP) &
		Eloquent (Active Record)  &
		Não utilizou \\
		\hline
		
		\citeonline{matos:2017} & 
		Spring (Java) &
		Hibernate (Data Mapper)  &
		Embutido no Spring \\
		\hline 
		
		\citeonline{matos:2017} & 
		Vaadin (Java) &
		Hibernate (Data Mapper)  &
		Não utilizou \\
		\hline 
		
		\citeonline{avelar:2018} & 
		Ninja (Java) &
		Hibernate (Data Mapper)  &
		Google Guice \\
		\hline 
		
		\citeonline{ferreira:2018} & 
		Wicket (Java) &
		Hibernate (Data Mapper)  &
		CDI \\ 
		\hline 
		
		\citeonline{ferreira:2018} & 
		Tapestry (Java) &
		Hibernate (Data Mapper)  &
		Embutido no Tapestry \\ 
		\hline 
		
		\citeonline{meirelles:2019} & 
		CodeIgniter (PHP) &
		Embutido no CodeIgniter (Active Record) &
		Não utilizou \\
		\hline 
		
		\citeonline{meirelles:2019} & 
		NodeJS (JavaScript) &
		Não utilizou  &
		Não utilizou \\ 
		\hline 
		
		\citeonline{guterres:2019} & 
		Play (Scala) &
		Slick (Padrão não identificado)  &
		Não utilizou \\ 
		\hline 
		
     	\citeonline{dalapicola:2021} & 
		AdonisJS e Quasar (Javascript) &
		Lucid (Active Record)  &
		Embutido no AdonisJS \\ 
		\hline 
		
		\citeonline{berger:2021} & 
		Yii (PHP) &
		Yii  (Data Mapper)  &
		Embutido no Yii \\ 
		\hline 
		
		\citeonline{berger:2021} & 
		Symfony (PHP)  &
		Doctrine (Active Record)  &
		Embutido no Symfony \\ 
		\hline 
	
	    \citeonline{gomes:2022} & 
		Angular e NodeJS(JavaScript) &
		TypeORM (Data Mapper)  &
		Embutido no Angular\\ 
		\hline 

        Este &
        Next.js (TypeScript) &
        Prisma (Data Mapper) &
        Não utilizou \\
        \hline
		
	\end{tabularx}
\end{table}
